%% ============================================================================
%% COMPUTATIONAL METHODOLOGY
%% ============================================================================
\section{Computational Methodology}
\label{sec:methodology}

WaveDL is designed as a modular, high-performance training engine that decouples \textit{physical modeling} (neural network architecture definitions) from \textit{computational infrastructure} (data loading, distributed synchronization, training loops). This separation of concerns enables researchers to focus on scientific innovation while leveraging battle-tested engineering components.

\subsection{System Architecture Overview}

The framework follows a layered architecture pattern:
\begin{enumerate}
    \item \textbf{Data Layer} (\texttt{utils/data.py}): Memory-mapped I/O, caching, and standardization
    \item \textbf{Model Layer} (\texttt{models/}): Registry pattern with abstract base class
    \item \textbf{Training Layer} (\texttt{train.py}): Accelerate-based distributed training loop
    \item \textbf{Utility Layer} (\texttt{utils/}): Metrics, visualization, and distributed primitives
\end{enumerate}

This modular design enables independent testing and replacement of components. For example, researchers can substitute the default StandardScaler with a RobustScaler for outlier-heavy datasets without modifying other layers.

% Figure: System architecture
\begin{figure*}[htbp]
\centering
\includegraphics[width=0.9\textwidth]{images/fig1_system_architecture.png}
\caption{WaveDL framework architecture showing the four-layer modular design. The Data Layer handles memory-mapped I/O and standardization, the Model Layer manages the registry pattern and model definitions, the Training Layer provides DDP and mixed-precision support, and the Utility Layer contains metrics and distributed synchronization primitives.}
\label{fig:system_architecture}
\end{figure*}

\subsection{Out-of-Core Data Management}

\subsubsection{Problem Formulation}

Let $\mathcal{D} = \{(\mathbf{x}_i, \mathbf{y}_i)\}_{i=1}^N$ be a dataset where $\mathbf{x}_i \in \mathbb{R}^{C \times H \times W}$ is a multi-channel 2D representation (e.g., dispersion image) and $\mathbf{y}_i \in \mathbb{R}^{K}$ is a vector of $K$ material properties. For typical guided wave applications with $N=10^5$ samples and $H=W=500$ pixels, the memory footprint is:
\begin{equation}
M_{total} = N \cdot C \cdot H \cdot W \cdot \text{sizeof(float32)} = 10^5 \cdot 1 \cdot 500 \cdot 500 \cdot 4 \approx 100~\text{GB}
\label{eq:memory_footprint}
\end{equation}

This exceeds the RAM of standard GPU nodes (typically 32--64~GB). Furthermore, even when RAM is sufficient, loading 100~GB into memory requires approximately 10--15 minutes at typical HDD speeds (100~MB/s), creating unacceptable startup latency for iterative development.

\subsubsection{Memory-Mapped Solution}

Memory mapping creates a virtual address space that mirrors the file on disk. The operating system's virtual memory manager handles page-level access:
\begin{itemize}
    \item \textbf{Page Fault on Access}: When the program accesses an unmapped region, a hardware interrupt triggers the OS to load the corresponding page from disk.
    \item \textbf{LRU Eviction}: Least Recently Used pages are evicted when physical RAM is exhausted.
    \item \textbf{Read-Ahead Prefetching}: Modern OSes predict access patterns and prefetch upcoming pages.
\end{itemize}

This mechanism provides $O(1)$ memory complexity (independent of $N$) and amortized $O(1)$ access time per sample due to caching.

% Figure: Memory-mapped pipeline
\begin{figure*}[htbp]
\centering
\includegraphics[width=0.9\textwidth]{images/fig2_memory_mapped_pipeline.png}
\caption{Memory-mapped data pipeline architecture. Large datasets stored on disk are mapped to virtual memory, enabling zero-copy access with $O(1)$ RAM usage regardless of dataset size. The OS handles page-level caching and prefetching transparently.}
\label{fig:memory_mapped}
\end{figure*}

\textbf{Algorithm 1: Memory-Mapped Dataset Initialization and Access}

\begin{lstlisting}[style=pseudocode,caption={Memory-Mapped Dataset Algorithm}]
PROCEDURE Initialize(file_path, shape, dtype):
    // Phase 1: Validate metadata (main process only)
    IF NOT file_exists(file_path):
        RAISE FileNotFoundError
    
    // DO NOT open memmap here - file descriptors not safe across fork()
    self.file_path <- file_path
    self.shape <- shape
    self.dtype <- dtype
    self.handle <- NULL  // Lazy initialization
    
PROCEDURE WorkerInit(worker_id):
    // Phase 2: Called by each DataLoader worker after fork()
    self.handle <- NULL  // Force re-open on first access
    
PROCEDURE GetItem(index):
    // Phase 3: Lazy handle acquisition
    IF self.handle IS NULL:
        self.handle <- mmap(self.file_path, mode='r', shape=self.shape, dtype=self.dtype)
    
    // Critical: copy() detaches tensor from memory-mapped buffer
    data <- self.handle[index].copy()
    RETURN tensor(data)
\end{lstlisting}

The \texttt{.copy()} operation in Phase 3 is essential. Without it, the returned tensor shares memory with the mapped buffer, causing:
\begin{enumerate}
    \item \textbf{Memory leaks}: PyTorch cannot track or free the external memory
    \item \textbf{Race conditions}: Multiple workers may access overlapping pages during data augmentation
    \item \textbf{Segmentation faults}: If the file is modified while tensors reference it
\end{enumerate}

\subsubsection{Complexity Analysis}

\textbf{Space Complexity}: The memory-mapped approach achieves $O(B)$ space complexity where $B$ is the batch size, compared to $O(N)$ for in-memory loading. This enables training on datasets of essentially unlimited size, constrained only by disk capacity.

\textbf{Time Complexity}: Individual sample access is $O(1)$ amortized. Cold-start access (first touch of a page) incurs disk I/O latency ($\sim$5--10~ms for HDD, $\sim$0.1~ms for NVMe SSD). Modern OS prefetching and sequential access patterns minimize cold-start overhead during training.

\textbf{I/O Throughput}: For batch size $B=128$ with $500 \times 500 \times 4$ bytes per sample, each batch requires $128 \times 10^6 = 128$~MB. On NVMe storage with 3~GB/s read bandwidth, theoretical maximum is $\sim$23 batches/second. In practice, GPU computation time ($\sim$50--100~ms/batch for typical CNNs) exceeds I/O time, making training compute-bound rather than I/O-bound.

\subsection{Distributed Data Parallel Synchronization}

\subsubsection{The Early Stopping Deadlock Problem}

In Distributed Data Parallel (DDP) training, each GPU runs an independent copy of the training loop. PyTorch's DDP synchronizes gradients via all-reduce operations during the backward pass. However, \textit{control flow decisions} (such as early stopping) are not automatically synchronized.

Consider the following failure scenario:
\begin{verbatim}
Time   | Rank 0 (Main)              | Rank 1
-------|----------------------------|---------------------------
t=100  | val_loss = 0.01            | val_loss = 0.01
t=101  | patience = 20 (stop!)      | patience = 19 (continue)
t=102  | exit training loop         | forward() -> wait for sync
t=103  | (terminated)               | DEADLOCK: waiting forever
\end{verbatim}

Rank 0 exits the training loop because its stopping condition is satisfied, but Rank 1's \texttt{forward()} call in the next iteration triggers gradient synchronization, which waits indefinitely for Rank 0's participation.

\subsubsection{Synchronized Termination Protocol}

WaveDL implements a broadcast-based synchronization protocol. Let $S_r \in \{0, 1\}$ be the local stopping decision at rank $r$:
\begin{equation}
S_r = \begin{cases}
    1 & \text{if } r = 0 \text{ and patience exhausted} \\
    0 & \text{otherwise}
\end{cases}
\end{equation}

The global stopping decision $S_{global}$ is computed via an all-reduce (MAX) operation:
\begin{equation}
S_{global} = \max_{r \in \{0, ..., P-1\}} S_r
\end{equation}

All ranks then check $S_{global}$ before proceeding to the next epoch. Since the all-reduce is a collective operation, all ranks must participate, ensuring no deadlock:
\begin{verbatim}
Time   | Rank 0 (Main)              | Rank 1
-------|----------------------------|---------------------------
t=100  | S_0 = 1 (stop!)            | S_1 = 0 (continue)
t=101  | all_reduce(S_0) -> 1       | all_reduce(S_1) -> 1
t=102  | S_global = 1 -> exit       | S_global = 1 -> exit
t=103  | (synchronized exit)        | (synchronized exit)
\end{verbatim}

% Figure: DDP synchronization
\begin{figure*}[htbp]
\centering
\includegraphics[width=0.9\textwidth]{images/fig3_ddp_synchronization.png}
\caption{DDP-safe early stopping protocol. All GPU ranks synchronize their stopping decision via an all-reduce operation before exiting, preventing deadlock scenarios where some ranks continue training while others terminate.}
\label{fig:ddp_sync}
\end{figure*}

\subsubsection{Metric Aggregation}

When computing global metrics across $P$ GPUs, na\"{i}vely averaging per-GPU averages produces incorrect results if batch sizes differ:

\textbf{Incorrect}:
\begin{equation}
\bar{L}_{wrong} = \frac{1}{P} \sum_{r=0}^{P-1} \bar{L}_r
\end{equation}

\textbf{Correct} (sum-of-sums / sum-of-counts):
\begin{equation}
\bar{L}_{correct} = \frac{\sum_{r=0}^{P-1} L_r^{sum}}{\sum_{r=0}^{P-1} N_r}
\end{equation}
where $L_r^{sum}$ is the total loss on rank $r$ and $N_r$ is the number of samples processed by rank $r$. WaveDL's \texttt{sync\_tensor} utility performs this reduction correctly.

\subsection{Modular Model Registry}

\subsubsection{Decorator-Based Registration}

To facilitate fair architecture comparison, WaveDL employs a \textbf{Factory Pattern} with decorator-based registration. This design:
\begin{enumerate}
    \item \textbf{Decouples model definition from training logic}: The training script uses string identifiers (e.g., \texttt{"ratenet"}) rather than class imports
    \item \textbf{Enables dynamic model selection}: Command-line arguments control which model is instantiated
    \item \textbf{Prevents circular imports}: The registry module loads before any model modules
\end{enumerate}

\textbf{Registry API}:
\begin{lstlisting}[caption={Model registration example}]
# Registration (in models/ratenet.py)
@register_model("ratenet")
@register_model("simplecnn")
class SimpleCNN(BaseModel):
    ...

# Usage (in train.py)
model = build_model(args.model, in_shape=(500, 500), out_size=2)
\end{lstlisting}

\subsubsection{BaseModel Abstract Class}

The \texttt{BaseModel} abstract class enforces a consistent interface:

\begin{table*}[htbp]
\centering
\caption{BaseModel interface methods}
\label{tab:basemodel}
\begin{tabular}{ll}
\toprule
\textbf{Method} & \textbf{Purpose} \\
\midrule
\texttt{\_\_init\_\_(in\_shape, out\_size, **kwargs)} & Constructor with required dimensions \\
\texttt{forward(x) $\to$ y} & Forward pass mapping input to output \\
\texttt{parameter\_summary() $\to$ dict} & Parameter count and memory footprint \\
\texttt{get\_optimizer\_groups(lr, wd) $\to$ list} & Differential learning rates for fine-tuning \\
\bottomrule
\end{tabular}
\end{table*}

This interface ensures that:
\begin{enumerate}
    \item All models are compatible with the training loop without modification
    \item Hyperparameter sweeps can iterate over models via string identifiers
    \item New architectures (e.g., Vision Transformers) can be added by implementing the interface
\end{enumerate}

% Figure: Registry pattern
\begin{figure*}[htbp]
\centering
\includegraphics[width=0.8\textwidth]{images/fig7_registry_pattern.png}
\caption{Decorator-based model registry pattern. Models inherit from BaseModel and register via decorators, enabling dynamic selection at runtime without modifying the training infrastructure.}
\label{fig:registry}
\end{figure*}

\subsection{Physics-Aware Metric Tracking}

\subsubsection{Target Standardization}

Neural network training benefits from standardized targets with zero mean and unit variance. Let $y_{ij}$ denote the $j$-th target of sample $i$. The standardization transform is:
\begin{equation}
\hat{y}_{ij} = \frac{y_{ij} - \mu_j}{\sigma_j}
\end{equation}
where $\mu_j$ and $\sigma_j$ are computed \textbf{on the training set only} to prevent data leakage. The fitted \texttt{StandardScaler} is saved with the checkpoint for use during inference.

\subsubsection{Physical Error Computation}

Reporting error in standardized units is meaningless to domain scientists. WaveDL automatically converts metrics to physical units:
\begin{equation}
\text{MAE}_j^{physical} = \sigma_j \cdot \text{MAE}_j^{standardized}
\end{equation}

For multi-target problems, errors are reported individually:
\begin{itemize}
    \item Target 0 (Thickness): $0.042 \pm 0.012$~mm
    \item Target 1 (Velocity): $3.1 \pm 0.8$~m/s
\end{itemize}

This enables direct comparison against engineering tolerances (e.g., ``Is thickness error within $\pm 0.1$~mm?'').

% Figure: Standardization flow
\begin{figure*}[htbp]
\centering
\includegraphics[width=0.9\textwidth]{images/fig14_standardization.png}
\caption{Data standardization and physical metric computation flow. Raw targets are standardized for training, then inverse-transformed for evaluation in physical units, enabling direct comparison with engineering tolerances.}
\label{fig:standardization}
\end{figure*}

\subsection{Implementation Details}

To ensure reproducibility and clarity, we provide the core implementation logic for the memory-mapped pipeline and distributed synchronization.

\textbf{Listing 1: Thread-Safe Memory-Mapped Dataset}
\begin{lstlisting}[caption={Thread-Safe Memory-Mapped Dataset}]
class MemmapDataset(Dataset):
    """
    Zero-copy dataset for large-scale physical simulations.
    
    Args:
        memmap_path (str): Path to the binary memmap file.
        shape (tuple): Dimensions of the complete dataset (N, C, H, W).
        dtype (str): Data type (default: 'float32').
    """
    def __init__(self, memmap_path, targets, shape, dtype='float32'):
        self.memmap_path = memmap_path
        self.targets = targets
        self.shape = shape
        self.dtype = dtype
        self.data = None  # Lazy initialization handle

    def __getitem__(self, index):
        # Lazy loading: Open file handle only on first access
        # This ensures each worker process has its own file descriptor
        if self.data is None:
            self.data = np.memmap(
                self.memmap_path, 
                dtype=self.dtype, 
                mode='r', 
                shape=self.shape
            )
        
        # .copy() is critical to detach from the memory buffer
        # and prevent shared memory race conditions during augmentation
        image = torch.from_numpy(self.data[index].copy())
        target = self.targets[index]
        
        return image, target
\end{lstlisting}

\textbf{Listing 2: DDP-Safe Early Stopping}
\begin{lstlisting}[caption={DDP-Safe Early Stopping}]
def broadcast_early_stop(should_stop: bool, accelerator) -> bool:
    """
    Synchronize early stopping decision across all GPU ranks.
    
    Args:
        should_stop (bool): Local decision from the main process.
        accelerator: HuggingFace Accelerate handler.
        
    Returns:
        bool: Harmonized decision (True if any rank decided to stop).
    """
    # Convert boolean to tensor for broadcasting
    device = accelerator.device
    stop_tensor = torch.tensor(int(should_stop), device=device)
    
    # Broadcast decision from Rank 0 to all other ranks
    if accelerator.num_processes > 1:
        # Use reduce to handle edge cases where non-rank-0 might trigger stop
        torch.distributed.all_reduce(stop_tensor, op=torch.distributed.ReduceOp.MAX)
        
    return stop_tensor.item() > 0
\end{lstlisting}
